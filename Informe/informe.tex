\documentclass[11pt]{article}
\usepackage[a4paper, margin=2.54cm]{geometry}
\usepackage[utf8]{inputenc}
% \usepackage[spanish, mexico]{babel}
\usepackage[article]{ragged2e}
\usepackage{textcomp}
\usepackage{amsmath}
\usepackage{amssymb}
\usepackage{amsfonts}
\usepackage{fancyvrb}
\usepackage{xcolor}
\usepackage{graphicx}
\usepackage{caption}
\usepackage{subcaption}
\usepackage{fancyvrb}
\usepackage{xcolor}
\setlength{\parindent}{0pt}

\renewcommand{\contentsname}{Contenidos}
\renewcommand\refname{Referencias}

\begin{document}
% \maketitle

\begin{titlepage}
    \centering
    {\bfseries\LARGE Universidad Nacional de Rosario \par}
    \begin{figure}[h!]
        \begin{center}
            \includegraphics[]{unr.png}
        \end{center}
      \end{figure}
    \vspace{1cm}
    {\scshape\Large Facultad de Ciencias Exactas, Ingeniería y Agrimensura \par}
    \vspace{3cm}
    {\scshape\Huge SAT-Solver \par}
    \vspace{3cm}
    {\itshape\Large Un evaluador para modelos de lógica temporal \par}
    \vfill
    {\Large Autor: \par}
    {\Large Natalia Mellino \par}
    \vfill
    {\Large Diciembre 2021 \par}
\end{titlepage}

\tableofcontents

\newpage

\section{Motivación}

\subsection{¿Por qué un DSL para Lógica Temporal?}

La principal motivación de la realización de este proyecto, no es más ni menos
que poder proveer una herramienta que pueda ser útil al momento de aprender y
poner en práctica los conocimientos adquiridos de la teoría de Lógica Temporal. \\

Al momento de realizar ejercicios o querer hacer alguna verificación sobre estos
modelos lógicos, nos encontramos con que no siempre podemos realizar estas tareas 
de manera rápida y simple como deseamos. Es por ello, que mi intención es poder 
facilitar en la medida que sea posible a cualquier persona que se encuentre en 
alguna de las situaciones mencionadas anteriormente.  

\section{Introducción}

\subsection{Un repaso de CTL}
El objetivo de esta sección no es dar una explicación exhaustiva de la teoría de 
lógica temporal, sino más bien, refrescar los conceptos que sean necesarios 
para un mayor entendimiento de cómo funciona este proyecto. Se asume que el lector
posee un cierto entendimiento de los conceptos básicos de Lógica Temporal. \\

Si bien en la Lógica Temporal hay muchas ramas y tópicos, en este proyecto nos 
centramos en lo que se conoce como CTL, o más bien, Computation Tree Logic. Este
es un tipo de lógica temporal en el que el se modela el tiempo como una estructura
de árbol en donde el futuro no está determinado, sino que hay distintos caminos 
en el futuro donde cualquiera de ellos puede ser el camino que se vaya a realizar \cite{logicbook}.

\subsubsection{Sintaxis y Semántica}

La \textbf{sintaxis} de CTL se puede definir de forma inductiva de la siguiente manera \cite{theory}:

\begin{align*}
  \phi :: = \top                       \; | \;  
            \bot                       \; & | \; 
            p                          \; | \; 
            \neg \phi                  \; | \;
            \phi \land \psi            \; | \;
            \phi \lor \psi             \; | \;
            \phi \rightarrow \psi      \; | \;
            \forall \bigcirc \phi      \; | \;
            \exists \bigcirc \phi      \; | \; \\
            & \forall [\phi \cup \psi] \; | \;
            \exists [\phi \cup \psi]   \; | \;
            \forall \square \phi       \; | \;
            \exists \square \phi       \; | \;
            \forall \diamondsuit \phi  \; | \;
            \exists \diamondsuit \phi 
\end{align*}

Un \textbf{Modelo} o \textbf{Sistema de transiciones} está formado por una tupla:
$(S, \rightarrow, I, L)$ donde:

\begin{itemize}
  \item $S$ es un conjunto finito de estados.
  \item $I$ es un conjunto de estados iniciales ($I \subseteq S$).
  \item $\rightarrow \; \subseteq S \times S$ es una relación de transición entre estados
  \item $L : S \rightarrow \mathcal{P}(AT)$ una función de etiquetado, donde a cada
        estado se le asigna un conjunto de proposiciones atómicas.
\end{itemize}

Ahora recordemos cómo era la \textbf{semántica} de estos operadores. Definimos 
la relación $\vDash$ por inducción en $\phi$:

\begin{itemize}
  \item $M,  s \nvDash \bot$
  \item $M, s \vDash p_i \iff p_i \in L(s)$
  \item $M, s \vDash \neg \phi \iff M, s \nvDash \phi$
  \item $M, s \vDash \phi \land \psi \iff M, s \vDash \phi $ y $M, s \vDash \psi$
  \item $M, s \vDash \forall \bigcirc \phi \iff $ para todo $s'$ tal que 
        $s \rightarrow s'$ se cumple que $M, s \vDash \phi$.
  \item $M, s \vDash \exists \bigcirc \phi \iff $ para algún $s'$ tal que
        $s \rightarrow s'$ se cumple que $M, s \vDash \phi$.
  \item $M, s \vDash \forall [\phi \cup \psi] \iff $ para cada traza
        $s_0 \rightarrow s_1 \rightarrow ...$ con $s_0 = s$ existe $j \in \mathbb{N}$
        tal que:
          \begin{itemize}
            \item $M, s_j \vDash \psi$
            \item $M, s_i \vDash \phi$ para todo $i < j$
          \end{itemize}
  \item $M, s \vDash \exists [\phi \cup \psi] \iff $ para alguna traza
        $s_0 \rightarrow s_1 \rightarrow ...$ con $s_0 = s$ existe $j \in \mathbb{N}$
        tal que:
          \begin{itemize}
            \item $M, s_j \vDash \psi$
            \item $M, s_i \vDash \phi$ para todo $i < j$
          \end{itemize}
\end{itemize}

Luego tenemos los \textbf{operadores derivados}:

\begin{itemize}
  \item $\forall \diamondsuit \phi = \forall [\top \cup \phi]  $
  \item $\exists \diamondsuit \phi = \exists [\top \cup \phi] $
  \item $\forall \square \phi = \neg \exists \diamondsuit \neg \phi $
  \item $\exists \square \phi = \neg \forall \diamondsuit \neg \phi $
\end{itemize}

Para los operadores $\rightarrow, \top$ se omitió su semántica ya que las mismas
se derivan a partir de los otros operadores ya existentes. Esto en realidad vale
también para varios de los operadores que definimos, en las próximas secciones
veremos como esto nos sirve para facilitar la implementación del evaluador.

\subsubsection{El Algoritmo SAT para \emph{model-checking}}

Recordemos rápidamente de qué se trataba el algoritmo SAT: dado un modelo y una 
fórmula CTL, devuelve el conjunto de estados del modelo que satisface dicha 
fórmula \cite{theory}. Esto es, en esencia, el objetivo de este proyecto. Para una fórmula
y un modelo, nuestro algoritmo se comporta de la siguiente manera:

\begin{itemize}
  \item $sat(\bot) = \emptyset$
  \item $sat(p_i) = \{s \in S \; | \; P_i \in L(s) \}$
  \item $sat(\neg \phi) = S - sat(\phi)$
  \item $sat(\phi \land \psi) = sat(\phi) \cap sat(\psi)$
  \item $sat(\phi \lor \psi) = sat(\phi) \cup sat(\psi)$
  \item $sat(\exists \bigcirc \phi) = pre_{\exists}(\phi)$
  \item $sat(\forall \bigcirc \phi) = pre_{\forall}(\phi)$
  \item $sat(\exists [\phi \cup \psi]) = exUntil(sat(\phi), sat(\psi))$
  \item $sat(\forall [\phi \cup \psi]) = forallUntil(sat(\phi), sat(\psi))$
  \item $sat(\forall \diamondsuit \phi) = inev(sat(\phi))$
  \item $sat(\exists \diamondsuit \phi) = exUntil(S, sat(\phi))$
\end{itemize}

Donde: \\

$preExists(Y) = \{s \in S \; | \; \exists s' : s \rightarrow s' \land s' \in Y\}$ \\
$preAll(Y) = \{s \in S \; | \; \forall s' : s \rightarrow s', s' \in Y\}$

\begin{verbatim}
  exUntil(X, Y):
    while (Y != Y u (X n preExists(Y))) do:
      Y <- Y u (X n preExists(Y))
    return Y
    
  forallUntil(X, Y):
    while (Y != Y u (X n preAll(Y))) do:
      Y <- Y u (X n preAll(Y))
    return Y

  inev(Y):
    while (Y != Y u preAll(Y)) do:
      Y <- Y u preAll(Y)
    return Y
\end{verbatim}

\section{Gramática}

\subsection{Sintaxis Concreta}
  Definimos la sintaxis concreta del DSL a partir de la siguiente gramática:

  \begin{Verbatim}[commandchars=\\\{\}]
    \textcolor{blue}{fields} ::= States '=' \textcolor{blue}{sts} ';' 
               Relations '=' \textcolor{blue}{rels} ';'
               Valuations '=' \textcolor{blue}{vals} ';'
               CTLExp '=' \textcolor{blue}{ctl} ';'

    \textcolor{blue}{vals} ::= '\{' '\}' | '\{' \textcolor{blue}{valuations} '\}'

    \textcolor{blue}{valuations} ::= \textcolor{blue}{valuation} | \textcolor{blue}{valuation} ',' \textcolor{blue}{valuations}

    \textcolor{blue}{valuation} ::= AT ':' \textcolor{blue}{sts}
    
    \textcolor{blue}{sts} ::= '[' \textcolor{blue}{states} ']' | '[' ']'

    \textcolor{blue}{states} ::= State | State ',' \textcolor{blue}{states}

    \textcolor{blue}{rels} ::= '[' ']' | '[' \textcolor{blue}{relations} ']'

    \textcolor{blue}{relations} ::= \textcolor{blue}{relation} | \textcolor{blue}{relation} ',' \textcolor{blue}{relations}

    \textcolor{blue}{relation} ::= '(' State ',' State ')'

    \textcolor{blue}{ctl} ::= AT                        
          | BT                        
          | TOP                       
          | '!' \textcolor{blue}{ctl}                  
          | \textcolor{blue}{ctl} '&' \textcolor{blue}{ctl}               
          | \textcolor{blue}{ctl} '|' \textcolor{blue}{ctl}              
          | \textcolor{blue}{ctl} '->' \textcolor{blue}{ctl}              
          | AX \textcolor{blue}{ctl}                     
          | EX \textcolor{blue}{ctl}                     
          | A '[' \textcolor{blue}{ctl} U  \textcolor{blue}{ctl} ']'                 
          | E '[' \textcolor{blue}{ctl} U  \textcolor{blue}{ctl} ']'                 
          | AF \textcolor{blue}{ctl}                     
          | EF \textcolor{blue}{ctl}                     
          | AG \textcolor{blue}{ctl}                     
          | EG \textcolor{blue}{ctl}                     
          | '(' \textcolor{blue}{ctl} ')'               

  \end{Verbatim}

  Para las fórmulas CTL, el \textbf{orden de precedencia} utilizado es el siguiente,
  comenzando desde los que tienen menos precedencia hasta los que más tienen.
  Dos o más operadores en el mismo item indican que la precedencia es la misma:

  \begin{itemize}
    \item Asocian a izquierda: $\land (\&) , \lor (|)$
    \item Asocian a izquierda: $\rightarrow (\texttt{->})$
    \item No asociativos: !, $\forall \bigcirc (AX), \exists \bigcirc (EX), \forall \diamondsuit (AF), \exists \diamondsuit (EF),
          \forall \square (AG), \exists \square (EG), \forall \cup (AU), \exists \cup (EU)$
  \end{itemize}

  Entonces por ejemplo: $p \land q \rightarrow r$ se asocia: $p \land (q \rightarrow r)$
  ya que $\rightarrow$ tiene más precedencia que $\land$. \\
  
  En los casos de los operadores que tienen la misma precedencia, se utiliza
  la asociación a izquierda, esto significa que si escribimos: $p \land q \lor r$, 
  este se asociará a la izquierda: $(p \land q) \lor r$. \\

  \textbf{Observación:} siempre es posible hacer uso de los parentesis para 
  explicitar el orden deseado de los operadores en una determinada fórmula.

\subsection{Notación}

Para más simplicidad al momento de parsear el modelo, cada símblolo
se representa de manera distinta en el código. A modo de referencia,
se puede consultar de forma rápida la siguiente tabla para entender
qué significa la notación utilizada a lo largo de todo el programa.

\begin{center}
  \begin{tabular}{ | c | c |}
   \hline
   \textbf{Operador} & \textbf{Traducción}  \\
   \hline 
   $\top$ & TOP  \\  
   \hline
   $\bot$ & BT   \\
   \hline
   $\neg \phi$ & !$\phi$   \\
   \hline
   $\phi \land \psi$ & $\phi \;  \&  \; \psi$   \\
   \hline
   $\phi \lor \psi$ & $\phi \; | \;  \psi$   \\
   \hline
   $\phi \rightarrow \psi$ & $\phi$ \texttt{->} $\psi$   \\
   \hline
   $\forall \bigcirc \phi$ & AX  $\phi$ \\
   \hline
   $\exists \bigcirc \phi$ & EX  $\phi$ \\
   \hline
   $\forall [\phi \cup \psi]$ & AU $\phi$  \\
   \hline
   $\exists [\phi \cup \psi]$ & EU $\phi$  \\
   \hline
   $\forall \square \phi$ & AG   $\phi$\\
   \hline
   $\exists \square \phi$ & EG   $\phi$\\
   \hline
   $\forall \diamondsuit \phi$ & AF $\phi$  \\
   \hline
   $\exists \diamondsuit \phi$ & EF $\phi$   \\
   \hline
  \end{tabular}
  \end{center} 

\section{Resolución}

\subsection{Parseo}

El modelo, junto con la fórmula a evaluar, es pasado en un archivo con
extensión \texttt{.sat} en el formato que explicita la sintaxis concreta
presentada en la sección anterior. Además, en la carpeta \texttt{test}
se pueden encontrar algunos tests a modo de ejemplo. \\

Se hizo uso de la herramienta \emph{Happy}, un generador de parsers
para Haskell, que dada una gramática específica, nos permite generar
un archivo Haskell con nuestro parser. Esta herramienta fue muy útil
ya que nos permitió de manera simple, especificar la asociatividad y
precedencia de los operadores. Además de que usando Happy, no 
tenemos que preocuparnos por la recursión a izquierda, lo cual
facilitó mucho el trabajo. \\

Podemos encontrar la implementación del parser en el archivo  \texttt{src/Parse.y}. \\

\textbf{Nota:} al pasar la fórmula CTL en el archivo de entrada es recomendable
utilizar espacios entre los operadores: AF, EF, AG, EG, AX, EX, y U para evitar
posibles errores de parseo. Por ejemplo, en vez de escribir \texttt{q\&AGp}, escribir:
\texttt{q \& AG p}

\subsection{Uso de las Mónadas}

Si vamos al archivo \texttt{src/Monads.hs} podemos ver que se hizo
uso de la \textbf{mónada State} para llevar el modelo (estados,
transiciones, valuaciones y fórmula) como estado global, ya que esto
nos permitió implementar el evaluador de forma más simple.

\subsection{Evaluación}

El evaluador es la parte principal del proyecto, es el que se encarga
de implementar y calcular nuestro algoritmo SAT. En \texttt{src/Eval.hs}
podemos ver su implementación utilizando mónadas. La misma no difiere
mucho de la explicación del algoritmo presentada al principio del 
informe, sigue exactamente la misma idea para implementar el algoritmo
y las funciones auxiliares.

\subsection{Otros Archivos}

\begin{itemize}
  \item En \texttt{src/CTL.hs} se definen las estucturas de datos
        utilizadas para representar nuestro modelo. A simple vista, 
        cuando uno da la fórmula CTL para el modelo, hace uso de todos
        los conectivos que fueron presentados anteriormente. Sin embargo, 
        recordemos que muchos de ellos en realidad no son necesarios ya que
        se pueden reescribir en función de otros (el ejemplo más claro es 
        el operador $\rightarrow$ que se puede representar utilizando 
        el $\lor$ y el $\neg$).  \\
        
        Por ello, en \texttt{src/CTL.hs} se definen dos estructuras distintas 
        para representar a una fórmula CTL: una representa a todos los operadores, 
        y la otra, sólo contiene los operadores necesarios para poder escribir
        cualquier fórmula. Entonces, luego del parseo lo que se hace es llamar
        a una función especial que se encarga de quitar todo este \emph{azucar
        sintáctico} y convertir el término parseado a uno que sólo usa los 
        operadores ``base". \\

        Sin embargo, mirando el código se puede ver que en el modelo se guardan
        ambas fórmulas, tanto la que tiene azúcar sintáctico como la que no lo
        tiene. Esto es así porque, para el evaluador utilizamos la fórmula 
        \emph{desugareada} y para el pretty printer se usa la que sí tiene
        azúcar sintáctico para poder mostrar el resultado en pantalla de una forma más
        elegante y legible.
  \item En \texttt{src/PPrint.hs} se implementó un simple pretty printer
        para mostrar los resultados en pantalla de manera más elegante.
  \item En \texttt{app/Main.hs} simplemente se encuentran las funciones que
        se encargan de:
        \begin{itemize}
          \item Leer y parsear el archivo de entrada que contiene el modelo.
          \item Llamar al evaluador con el modelo dado.
          \item Tomar el resultado devuelto y mostrarlo en pantalla utilizando 
          las funciones de \texttt{src/PPrint.hs}.
        \end{itemize}
        En el main, también se realiza cierto chequeo sobre el modelo, para
        advertir al usuario sobre los errores que se puedan encontrar.
\end{itemize}

\section{Instalación y Uso}

Para comenzar a usar el evaluador basta con lo siguiente:

\begin{itemize}
  \item Clonar el repositorio: \\
        \texttt{\$ git clone https://github.com/natimellino/SAT-Solver.git}
  \item Una vez dentro del repositorio, correr: \\
        \texttt{\$ stack setup}
  \item Para compilar corremos: \\
        \texttt{\$ stack build}
  \item Finalmente, para ejecutar nuestro evaluador con un modelo: \\
        \texttt{\$ stack exec TP-Final-exe "/path/to/file.sat"}
\end{itemize}

\section{Trabajo a Futuro}

\begin{itemize}
  \item Implementar un mejor manejo de errores en el parser, que nos permita 
  identificar con más precisión cuáles son los errores de sintaxis en el archivo
  de entrada.
  \item Hacer una versión interactiva del programa, esta nos podría permitir
  modificar el modelo pasado como argumento sin tener que volver a ejecutar el
  programa. Esto se realizaría agregando funciones que permitan modificar el 
  estado de nuestra mónada que es el que lleva el modelo.
  \item Una vez que se tenga una versión interactiva del programa se pueden
  agregar más comandos que el de simplemente evaluar un modelo, podría ser por
  ejemplo imprimir en pantalla la fórmula o el modelo (mejorando nuestro pretty
  printer para mostrar resultados más elegantes); y también proveer los comandos
  que nos permiten modificar nuestro modelo de entrada, o bien, pasar un modelo nuevo,
  haciendo posible lo mencionado en el apartado anterior.
\end{itemize}

\newpage

\begin{thebibliography}{9}
  \bibitem{logicbook}
  Michael Huth, Mark Ryan - \emph{Logic in computer science modelling and reasoning 
  about systems}, Cambridge University Press (2004).
  \bibitem{theory}
  Diapositivas de teoría de \emph{Lógica Temporal} de Dante Zanarini, FCEIA (año 2019).
\end{thebibliography}

\end{document}